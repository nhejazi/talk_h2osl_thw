\documentclass[12pt,t]{beamer}
\usepackage{graphicx}
\setbeameroption{hide notes}
\setbeamertemplate{note page}[plain]
\usepackage{listings}

\input{header.tex}

%%%%%%%%%%%%%%%%%%%%%%%%%%%%%%%%%%%%%%%%%%%%%%%%%%%%%%%%%%%%%%%%%%%%%%
% end of header
%%%%%%%%%%%%%%%%%%%%%%%%%%%%%%%%%%%%%%%%%%%%%%%%%%%%%%%%%%%%%%%%%%%%%%

% title info
\title{TITLE GOES HERE}
\author{\href{http://nimahejazi.org}{Nima Hejazi}}
\institute{Division of Biostatistics \\ University of California, Berkeley}
\date{\href{http://nimahejazi.org}{\tt \scriptsize \color{foreground} nimahejazi.org}
\\[-4pt]
\href{https://github.com/nhejazi}{\tt \scriptsize \color{foreground}
github.com/nhejazi}
\\[-4pt]
\href{https://twitter.com/nshejazi}{\tt \scriptsize \color{foreground} @nshejazi}
\\[2pt]
}


\begin{document}

% title slide
{
\setbeamertemplate{footline}{} % no page number here
\frame{
  \titlepage

  \vfill \hfill \includegraphics[height=6mm]{Figs/cc-zero.png} \vspace*{-1cm}

  \note{These are slides for a talk I've given a whole bunch of times,
    most recently at for the Computational Biology and Biostatistics
    Summer Research Program (CBB) at UW-Madison on 7 June 2016.

    Source: {\tt https://github.com/kbroman/Talk\_ReproRes} \\
    Slides: {\tt http://bit.ly/CBB2016\_nonotes} \\
    With notes: {\tt http://bit.ly/CBB2016\_wnotes}
}
}
}


\begin{frame}[fragile,c]{}

\begin{center}
\begin{minipage}[c]{9.3cm}
\begin{semiverbatim}
\lstset{basicstyle=\normalsize}
\begin{lstlisting}[linewidth=9.3cm]
 Karl -- this is very interesting,
 however you used an old version of
 the data (n=143 rather than n=226).

 I'm really sorry you did all that
 work on the incomplete dataset.

 Bruce
\end{lstlisting}
\end{semiverbatim}
\end{minipage}
\end{center}

\note{This is an edited version of an email I got from a collaborator,
  in response to an analysis report that I had sent him.

  I try to always include some brief data summaries at the start of
  such reports. By doing so, he immediately saw that I had an old
  version of the data.

  Because I'd set things up carefully, I could just substitute in the
  newer dataset, type ``{\tt make}'', and get the revised report.

  This is a reproducibility success story. But it took me a long
  time to get to this point.
}
\end{frame}


\begin{frame}[c]{}
\centering
{\Large The results in Table 1 don't seem to \\[12pt]
correspond to those in Figure 2.}

\note{My computational life is not entirely rosy. This is the sort of
  email that will freak me out.}
\end{frame}


\begin{frame}[c]{}
\centerline{\Large In what order do I run these scripts?}

\note{Sometimes the process of data file manipulation and data
  cleaning gets spread across a bunch of scripts that need to be
  executed in a particular order. Will I record this information? Is
  it obvious what script does what?}
\end{frame}


\begin{frame}[c]{}
\centerline{\Large Where did we get this data file?}

\note{Record the provenance of all data or metadata files.}
\end{frame}


\begin{frame}[c]{}
\centerline{\Large Why did I omit those samples?}

\note{I may decide to omit a few samples. Will I record {\nhilit why}
  I omitted those particular samples?}
\end{frame}


\begin{frame}[c]{}
\centerline{\Large How did I make that figure?}

\note{Sometimes, in the midst of a bout of exploratory data analysis,
  I'll create some exciting graph and have a heck of a time
  reproducing it afterwards.}
\end{frame}


\begin{frame}[c]{}
\centerline{\Large ``Your script is now giving an error."}

\note{It was working last week. Well, last month, at least.

How easy is it to go back through that script's history to see where
and why it stopped working?}
\end{frame}


\begin{frame}[c]{}
\centerline{\Large ``The attached is similar to the code we used."}

\note{From an email in response to my request for code used for a
  paper.}
\end{frame}


\begin{frame}[c]{7. Use version control (git/GitHub)}

\only<1>{\addtocounter{framenumber}{-1}}

\vspace*{3mm}

\centering

%\only<1|handout 0>{\figh{Figs/example_history}{0.80}}
%\only<2>{\figh{Figs/example_commit}{0.80}}

\note{
  git has a steep learning curve, but ultimately I think you'll find
  it really helpful.

  The big selling point is in collaboration: merging changes from
  collaborators, and keep your work synchronized.

  Longer term, there's great value in having the entire history of
  changes to your project. If something stops working, you can go
  back to any point in that history to see when it stopped working and
  why.

  With git, you can also work on new features or analyses without fear
  of breaking the parts that are currently working well.
}
\end{frame}


\begin{frame}{8. License your software}

\vspace{60pt}

\centerline{\large Pick a license, any license}

\vspace{18pt}

\hfill
{\textendash} \href{http://blog.codinghorror.com/pick-a-license-any-license/}{Jeff Atwood}

\note{
  If you don't pick a license for your software, no one else can use it.

  So if you want to distribute your code so that others can reproduce
  your analyses, you need to pick a license, any license.

  I choose between the MIT license and the GPL.

  Don't use the Creative Commons licenses for code. But feel free to
  use them for other things.
}
\end{frame}


\begin{frame}[c]{Summary}

  \begin{enumerate}
  \itemsep12pt
  \item Organize your data \& code
  \item Everything with a script
  \item Automate the process (GNU Make)
  \item Turn scripts into reproducible reports
  \item Turn repeated code into functions
  \item Create a package/module
  \item Use version control (git/GitHub)
  \item Pick a license, any license
  \end{enumerate}

  \note{
    It's always good to include a summary.
}
\end{frame}


\begin{frame}[c]{}

\Large

Slides: \href{http://bit.ly/CBB2016_wnotes}{\tt bit.ly/CBB2016\_wnotes} \quad
\includegraphics[height=5mm]{Figs/cc-zero.png}

\vspace{10mm}

\href{http://nimahejazi.org}{\tt nimahejazi.org}

\vspace{10mm}

\href{https://github.com/nhejazi}{\tt github.com/nhejazi}

\vspace{10mm}

\href{https://twitter.com/nhejazi7}{\tt @nhejazi7}


\note{
  Here's where you can find me, as well as the slides for this talk.
}
\end{frame}


\end{document}
